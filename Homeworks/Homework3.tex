\documentclass[12pt,letterpaper, onecolumn]{exam}
\usepackage{amsmath}
\usepackage{amssymb}
\usepackage{graphicx}
\usepackage[lmargin=71pt, tmargin=1.2in]{geometry}  %For centering solution box

% \chead{\hline} % Un-comment to draw line below header
\thispagestyle{empty}   %For removing header/footer from page 1

\begin{document}

\begingroup  
    \centering
    \LARGE STATS 212\\
    \LARGE Homework#3 \\
    \large \today\\[0.5em]
    \large Samuel Molero\par
    \large samueljosemolero@tamu.edu\par
    \large Section: 501\par
\endgroup
\rule{\textwidth}{0.4pt}
\pointsdroppedatright   %Self-explanatory
\printanswers
\renewcommand{\solutiontitle}{\noindent\textbf{Ans:}\enspace}   %Replace "Ans:" with starting keyword in solution box

\begin{questions}

    \question Q1?
    \begin{solution}
            
    \end{solution}
    
    \question Q2?
    \begin{solution}
        \begin{parts}
            \part \includegraphics[width=0.7\textwidth]{Homeworks/partAHW3.jpg}
            \part
            \begin{verbatim} 
    dta2 <- read.table("SleepRem.txt", header = TRUE, sep = "")
    attach(dta2)
    fit <- aov(values ~ as.factor(ind), data = dta2)
    anova(fit)
    \end{verbatim}
    Using the above code snippet we get the following result:
    \begin{verbatim}
Response: values
        Df  Sum Sq  Mean Sq  F value  Pr(>F)    
find     3  5881.7 1960.58  21.093 8.322e-06 
Residuals 16 1487.1   92.95                      
    \end{verbatim}
    Given that is confirmed that the p-value is $8.322e-06$ the null hypothesis should be rejected.
    \part 
    \begin{verbatim}
      #To test variance
      anova(aov(resid(aov(values ~ ind))**2 ~ ind))
    \end{verbatim}
    Given that the p-value is 0.621. This suggests that the assumption of equal
    variance is approximately valid.
    \begin{verbatim}
    #to test stability
    shapiro.test(resid(aov(values ~ ind)))
    \end{verbatim}
     p-value = 0.1285 suggests that the normality assumption
    approximately holds.
    \end{parts}
    \begin{verbatim}
    fit = aov(values ~ find)
    boxplot(resid(fit))
    plot(fit, which=2)
    \end{verbatim}
    The Code snippets creates the following graphs:\\
    \center
    \includegraphics[width=0.7\textwidth]{Homeworks/Graph1HW3.png}\\
    \includegraphics[width=0.7\textwidth]{Homeworks/Graph2HW3.png}
    plots also suggest that the normality assumption is approximately satisfied, in agreement with the Shapiro-Wilk test p-value.
    \end{solution}

    \pagebreak %Not necessary
    
    
\end{questions}
\end{document}